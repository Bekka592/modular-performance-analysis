\mysection{Discussion}\label{discussion}

\mysubsection{Advantages and Disadvantages of using a Formal Analysis Method}
In contrast to simulation based, probability based or holistic analysis methods for embedded real-time systems, formal methods can guarantee the correctness of their results~\cite{wan:06}.
The analysis will surely reveal the worst case bounds of system properties such as end-to-end delays.
Therefore, it is guaranteed that any system architecture accepted by the analysis, also fulfills all system requirements in reality~\cite{wan:06}.
On the other hand, system analysis using formal methods often leads to very pessimistic and in general not tight bounds, due to
a lack of details to include in the analysis and limited modeling capacities~\cite{wan:05}~\cite{cho:08}.
Using formal methods, we can only compute hard upper and lower bounds to the real system characteristics, which makes this method only suitable to analyze hard real-time systems only.
A system architecture might be rejected due to bad worst case guaranties, even if they are never actually reached
in execution and therefore the system in fact fulfills all requirements\cite{wan:06}.

Other significant advantages of the formal analysis are the fast analysis run-time, as well as the low set up effort for new architectures~\cite{wan:06}.
Similar models can be easily and quickly constructed and compared, as this merely involves reconnecting the event and service flows between the components.
In this way the technique is sufficiently productive even with a really high time-to-market pressure in the domain of embedded real-time systems and 
the interactive nature of the design process of such systems~\cite{ver}~\cite{hoo}.

Besides that, formal analysis methods are not suitable for systems with a dynamic and complex interactions, as this architecture can not be incorporated into the model properly~\cite{wan:06}.

\mysubsection{Advantages of Using MPA}

In contrast to many other formal analysis methods of embedded systems, the modular performance
analysis relies neither on standard arrival patterns of events, nor on classical scheduling~\cite{pha}.
In general the MPA can work really efficiently with only really few and abstract data about the system.
The low demand on detailed information, makes MPA a lightweight analysis method~\cite{wan:06}.
Furthermore, this leads to a high degree of generality and modularity, as well as an easy analyzability of system performance models~\cite{wan:06}.

The MPA and RTC framework is easy to use, especially when working with the \href{https://www.mpa.ethz.ch/}{RTC Toolbox for Matlab}.
It is a compositional method which simplifies the analysis of heterogeneous systems a lot.
In addition, MPA can represent the majority of different event arrival patterns, whether arbitrary or not~\cite{cho:08}.

\mysubsection{Disadvantages and Limitations of Using MPA}

As we are abstraction from the time domain to the time interval domain, it is
difficult to accurately exploit implicit timing correlations between event arrivals on different event streams within the MPA framework~\cite{wan:06}.

In MPA and RTC it is impossible to model state-dependent behavior, as state information cannot be modeled naturally. 
For example, scheduling policies, that depend on the fill-level of buffers just cannot be implemented in the framework of MPA.\
Yet, thanks to~\cite{pha} we can interface RTC curves with state-based models using automata and hence also model states in this new extended framework.