\mysection{Conclusion}\label{conclusion}

The framework of modular performance analysis proofs to be a powerful analyzing tool for hard distributed embedded real-time systems, 
that is capable of doing fast evaluations on system structures.
As shown in \autoref{discussion} using MPA with RTC has some major advantages and is useful for various contexts, 
especially for the use in early design phases of an embedded system to make important design decisions, as MPA can deal with only few and abstract information about the system~\cite{wan:06}.

The real-time calculus framework furthermore empowered MPA by formal analysis methods and efficient computation strategies for obtaining correct and accurate performance analysis results.
Even though the computed bounds on system performance characteristics, such as end-to-end delays and buffer requirements, are often quite pessimistic, they provide guaranteed hard bounds~\cite{wan:06}.

Although there are some limits to MPA and RTC, they both are really valuable approaches to the formal analysis during the design process of real-time embedded systems.
Therefore, they are still subjects of ongoing research.
Hence, there is a good chance, that some problems described in \autoref{discussion} will be solved at least partially within the next few years.

If enough information about the system and time to do the analysis are provided, it might also be an option to 
use MPA combined with a simulation based method for obtaining even better insights into the system~\cite{wan:06}.