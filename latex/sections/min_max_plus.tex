\section{The Min/Max-Plus Calculus}\label{minMaxPlus}

Real-time calculus uses the min/max-plus algebra for computational purposes.
Instead of the usual algebraic structure of \((\mathbb{R},+,\cdot)\), we are now dealing with
\((\mathbb{R}\cup\infty,+,\cdot)\) as an algebraic structure. 
Luckily we do not need to deep dive into the characteristics of the algebra here, 
knowing and understanding the following operations on functions \(f\) and \(g\) is sufficient for being able to
use RTC as a calculation tool:

\begin{align*}
    \text{Min-Plus Convolution:} \\
    (f \otimes g)(\Delta t) &= \inf _{0 \leq \lambda \leq \Delta t} \left\{ f (\Delta t - \lambda) + g (\lambda) \right\} \\
    \text{Min-Plus Deconvolution:} \\
    (f \oslash g)(\Delta t) &= \sup _{\lambda \geq 0} \left\{ f (\Delta t + \lambda) - g (\lambda) \right\} \\
    \text{Max-Plus Convolution:} \\
    (f \overline{\otimes} g)(\Delta t) &= \sup _{0 \leq \lambda \leq \Delta t} \left\{ f (\Delta t - \lambda) + g (\lambda) \right\} \\
    \text{Max-Plus Deconvolution:} \\
    (f \overline{\oslash} g)(\Delta t) &= \inf _{\lambda \geq 0} \left\{ f (\Delta t + \lambda) - g (\lambda) \right\}
\end{align*}

Proofs and more detailed information on the min/max-plus algebra can be found in~\cite{bou}.
These equations can be quite complicated to compute,
but the \href{https://www.mpa.ethz.ch/}{RTC Toolbox for Matlab} is here to take care of them.
This toolkit is a great help when it comes to performing MPA for any kind of use case scenarios.

%As a reminder: The Supremum of a set of elements is defined as the set's least upper bound.
%In case it is within the set, the Supremum can be called a maximum.
%In an analogue way the Infimum of a set is defined as its greatest lower bound.